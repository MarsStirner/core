 \documentclass[a4paper,8pt]{report} %размер бумаги устанавливаем А4, шрифт 8пунктов
 \usepackage{kmanual} % собственный стиль
 \usepackage{makeidx}
 \usepackage{longtable, tabu}
 \renewcommand{\familydefault}{\sfdefault}
 \makeindex

 \makeatletter

 \makeatother

 \begin{document}
 {\small
  \begin{titlepage}
    \newpage

    \begin{center}
    \vspace*{\fill}
    \hbox{%
    \vrule width 4pt\hspace{2em}\parbox{1\textwidth}%
    {\vspace*{0.5em}\raggedright{\Large{\textbf{ФЕДЕРАЛЬНАЯ ТИПОВАЯ \\ МЕДИЦИНСКАЯ ИНФОРМАЦИОННАЯ СИСТЕМА}}}

    \vspace*{1em}
    \Large{\textbf{Описание процесса движения пациентов по отделениям}}
    }}
    \end{center}

    \vspace{\fill}

    \begin{flushleft}
    Дата создания:  14.04.14 \\
 %   \vspace{1em}
    Последнее обновление:  14.04.14\\
 %   \vspace{1em}
    Версия:  0.1a\\
    \end{flushleft}
    \clearpage
    \end{titlepage}
 В основе реализации движений пациентов по отделениям лежит таблица Action (Таблица \ref{db.ActionTable}).
 {\scriptsize  {\tabulinesep=1.2mm
 \begin{longtabu} to \textwidth {|X[1,c]|X[1,l]|X[1.5,l]|}
 \caption{Описание полей таблицы Action} \label{db.ActionTable}\\
 \hline
 Поле                   & Описание                                          & MySQL описание \\                             \hline
 id                     & Идентификатор действия                            & int(11) NOT NULL AUTO\_INCREMENT \\           \hline
 createDatetime         & Дата создания записи                              & datetime NOT NULL \\                          \hline
 createPerson\_id       & Автор записи, ссылка на таблицу Person            & int(11) DEFAULT NULL \\                       \hline
 modifyDatetime         & Дата изменения записи                             & datetime NOT NULL \\                          \hline
 modifyPerson\_id       & Автор изменения записи, ссылка на таблицу Person  & int(11) DEFAULT NULL \\                       \hline
 deleted                & Отметка удаления записи                           &  tinyint(1) NOT NULL DEFAULT \\               \hline
 actionType\_id         & Тип действия, ссылка на таблицу ActionType        & int(11) NOT NULL \\                           \hline
 event\_id              & Событие, к которому относистся действие, ссылка на
                            таблицу Event                                   & int(11) DEFAULT NULL \\                       \hline
 idx                    & Индекс в списке событий (для сортировки в списке) & NOT NULL DEFAULT '0' \\                       \hline
 directionDate          & Дата назначения                                   & datetime DEFAULT NULL \\                                                                 \hline
 status                 & Статус выполнения: 0-Начато, 1-Ожидание,
                            2-Закончено, 3-Отменено                         & tinyint(4) NOT NULL \\                        \hline
 setPerson\_id          & Назначивший, ссылка на таблицу Person             & int(11) DEFAULT NULL \\                       \hline
 isUrgent               & Является срочным                                  & int(1) NOT NULL DEFAULT '0' \\                \hline
 begDate                & Дата начала работы                                & datetime DEFAULT NULL \\                      \hline
 plannedEndDate         & Плановая дата выполнения                          & datetime NOT NULL \\                          \hline
 endDate                & Дата окончания работы                             & datetime DEFAULT NULL \\                      \hline
 note                   & Примечания                                        & text NOT NULL \\                              \hline
 person\_id             & Исполнитель, ссылка на таблицу Person             & int(11) DEFAULT NULL \\                       \hline
 office                 & Кабинет                                           & varchar(16) NOT NULL \\                       \hline
 amount                 & Количество                                        & double NOT NULL \\                            \hline
 uet                    & УЕТ (Условные Единицы Трудозатрат)                & double DEFAULT '0' \\                         \hline
 expose                 & Выставлять счёт                                   & int(1) NOT NULL DEFAULT '1' \\                \hline
 payStatus              & Флаги финансирования                              & int(11) NOT NULL \\                           \hline
 account                & Флаг Считать                                      & tinyint(1) NOT NULL \\                        \hline
 finance\_id            & Тип финансирования, ссылка на таблицу rbFinance   & int(11) DEFAULT NULL \\                       \hline
 prescription\_id       & Назначение, ссылка на талицу Action               & int(11) DEFAULT NULL \\                       \hline
 takenTissueJournal\_id & Журнал забора тканей, ссылка на
                            таблицу TakenTissueJournal                      & int(11) DEFAULT NULL \\                       \hline
 contract\_id           & Договор, ссылка на таблицу Contract               & int(11) DEFAULT NULL \\                       \hline
 coordDate              & Дата и время согласования                         & datetime DEFAULT NULL \\                      \hline
 coordAgent             & Сотрудник ЛПУ, согласовавший действие             & varchar(128) NOT NULL DEFAULT '' \\           \hline
 coordInspector         & Представитель плательщика (сотрудник СМО),
                            согласовавший действие                          & varchar(128) NOT NULL DEFAULT '' \\           \hline
 coordText              & Текст согласования                                & tinytext NOT NULL \\                          \hline
 hospitalUidFrom        &                                                   & varchar(128) NOT NULL DEFAULT '0' \\          \hline
 pacientInQueueType     &                                                   & tinyint(4) DEFAULT '0' \\                     \hline
 AppointmentType        & amb - амбулаторный пациент, hospital - пациент из
                            стационара (прием в стационаре), polyclinic -
                            пациент из стационара (прием в поликлинике),
                            diagnostics - диагностика                       & enum('0', 'amb',
                                                                                'hospital', 'polyclinic', 'diagnostics',
                                                                                'portal', 'otherLPU') \\                    \hline
 version                &                                                   & int(11) NOT NULL DEFAULT '0' \\               \hline
 parentAction\_id       & Родительское действие, ссылка на таблицу Action   & int(11) DEFAULT NULL \\                       \hline
 uuid\_id               &                                                   & int(11) NOT NULL DEFAULT '0' \\               \hline
 dcm\_study\_uid        & Внешний идентификатор записи в системе
                            DICOM Архив                                     & varchar(50) DEFAULT NULL \\                   \hline
 \end{longtabu}
}}

 Движение пациента описывается следующими типами действий: \\
 \begin{table}[h]
 \begin{tabular}{|c|c|}
 \hline
 Имя типа & flatcode \\
 \hline
 Поступление & received \\
 Движение & moving \\
 Выписка & leaved \\
 \hline
 \end{tabular}
 \end{table}

 И происходит в следующем порядке:
 \begin{enumerate}
  \item Создание действия типа "Поступление", направляющего пациента в отделение (как правило, приемное)
  \item Установка у действия типа "Поступление" свойства "Напавлен в отделение" (orgStructDirection). Тем самым, мы направляем пациента в следующее отделение.
  \item Создание действия типа "Движение", которое характеризует размещение пациента на койке.
  \item Установка у действия типа "Движение" свойства "Переведен в отделение" (orgStructTransfer). Тем самым, мы направляем пациента в следующее отделение.
  \item Создание действия типа "Движение", которое характеризует размещение пациента на койке.
  \item Закрытие последнего действия типа "Движение".
  \item Создание действия типа "Выписка".
\end{enumerate}
 }
 \end{document}
